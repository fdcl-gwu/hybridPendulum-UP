\documentclass[10pt]{article}

\usepackage{amssymb,amsmath,amsthm}
\usepackage{bm}
\usepackage{graphicx,subcaption}
\usepackage[letterpaper, top=1in, left=1in, right=1in, bottom=1in]{geometry}

\newtheorem{definition}{Definition}
\newtheorem{theorem}{Theorem}
\newtheorem{lemma}{Lemma}
\newtheorem{remark}{Remark}

\newcommand{\SO}{\ensuremath{\mathrm{SO}(3)}}
\newcommand{\tr}[1]{\ensuremath{\mathrm{tr}\left( #1 \right)}}
\newcommand{\abs}[1]{\ensuremath{\left| #1 \right|}}
\newcommand{\diff}[1]{\mathrm{d}#1}
\newcommand{\vect}[1]{\ensuremath{\mathrm{vec}\left[ #1 \right]}}

\newcommand{\liediff}{\mathfrak{d}}
\newcommand{\dft}{\mathcal{F}}
\newcommand{\real}{\ensuremath{\mathbb{R}}}
\newcommand{\sph}{\ensuremath{\mathbb{S}}}
\newcommand{\diag}{\ensuremath{\mathrm{diag}}}

\begin{document}

The equation of motion for the 3D pendulum is
\begin{align*}
	\dot{R} &= R\hat{\Omega}, \\
	\dot{\Omega} &= -J^{-1} \left( \Omega\times J\Omega + mg\rho\times R^Te_3 \right).
\end{align*}
Let $p(R,\Omega,t)$ denote the probability density of $(R,\Omega)$ at time $t$.
Then the Fokker-Planck equation of $p$ is
\begin{align} \label{eqn:FP}
	\frac{\partial p(R,\Omega,t)}{\partial t} &= -\sum_{i=1}^{3}\liediff_i [\Omega_ip(R,\Omega,t)] + \sum_{i=1}^3 \frac{\partial \Big[ \big(J^{-1}(\Omega\times J\Omega + mg\rho\times R^Te_3)\big)_i p(R,\Omega,t) \Big]}{\partial \Omega_i} \\
	&= -\sum_{i=1}^2 \Omega_i \liediff_ip(R,\Omega,t) + \left( \sum_{i=1}^3 \frac{\partial \big[J^{-1}(\Omega\times J\Omega)\big]_i}{\partial \Omega_i} \right)p(R,\Omega,t) \nonumber \\
	&\qquad + \sum_{i=1}^3 \left( \big(J^{-1}(\Omega\times J\Omega + mg\rho\times R^Te_3)\big)_i \frac{\partial p(R,\Omega,t)}{\partial \Omega_i} \right). \nonumber
\end{align}
By direct calculation (this is equivalent to $\mathrm{div}X = 0$), it can be shown that
\begin{align*}
	\sum_{i=1}^3 \frac{\partial \big[J^{-1}(\Omega\times J\Omega)\big]_i}{\partial \Omega_i} = 0.
\end{align*}
Thus the Fokker-Planck equation can be simplified as
\begin{align} \label{eqn:FP2}
	\frac{\partial p(R,\Omega,t)}{\partial t} = -\sum_{i=1}^2 \Omega_i \liediff_ip(R,\Omega,t) + \sum_{i=1}^3 \left( \big(J^{-1}(\Omega\times J\Omega + mg\rho\times R^Te_3)\big)_i \frac{\partial p(R,\Omega,t)}{\partial \Omega_i} \right).
\end{align}
Note that this equation is actually
\begin{align*}
	\frac{\diff{p(R,\Omega,t)}}{\diff{t}} = \frac{\partial p(R,\Omega,t)}{\partial t} + \mathcal{L}_X p(R,\Omega,t) = 0,
\end{align*}
where $X = \sum_{i=1}^3 \left( \Omega_i\liediff_i - \big[ J^{-1}(\Omega\times J\Omega + mg\rho\times R^Te_3) \big]_i \partial/\partial\Omega_i \right)$ is the vector field given by the equation of motion written in local coordinates, such that any trajectory of $(R,\Omega)$ is an integral curve of $X$.

Unfortunately, compared with \eqref{eqn:FP}, \eqref{eqn:FP2} does not improve computational efficiency.
Their only difference is whether apply derivative before or after taking product.
When calculating the Fourier coefficient, this dose not change the amount of computation, but only the sequence.

\end{document}

