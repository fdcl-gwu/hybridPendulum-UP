\documentclass[10pt]{article}

\usepackage{amssymb,amsmath,amsthm}
\usepackage{bm}
\usepackage{graphicx,subcaption}
\usepackage[letterpaper, top=1in, left=1in, right=1in, bottom=1in]{geometry}

\newtheorem{definition}{Definition}
\newtheorem{theorem}{Theorem}
\newtheorem{lemma}{Lemma}
\newtheorem{remark}{Remark}

\newcommand{\SO}{\ensuremath{\mathrm{SO}(3)}}
\newcommand{\tr}[1]{\ensuremath{\mathrm{tr}\left( #1 \right)}}
\newcommand{\abs}[1]{\ensuremath{\left| #1 \right|}}
\newcommand{\diff}[1]{\mathrm{d}#1}
\newcommand{\vect}[1]{\ensuremath{\mathrm{vec}\left[ #1 \right]}}

\newcommand{\liediff}{\mathfrak{d}}
\newcommand{\dft}{\mathcal{F}}
\newcommand{\real}{\ensuremath{\mathbb{R}}}
\newcommand{\sph}{\ensuremath{\mathbb{S}}}
\newcommand{\diag}{\ensuremath{\mathrm{diag}}}

\title{\vspace{-4ex}\textbf{Fokker-Planck Equation for the Log-density Function\vspace{-4ex}}}
\date{}

\begin{document}

\maketitle

The usual Fokker-Planck equation for the density function $f(x,t)$ is
\begin{align*}
	\frac{\partial f}{\partial t} = -\sum_{i=1}^n \frac{\partial (\mu_if)}{\partial x_i} + \sum_{i,j=1}^n \frac{\partial^2 (D_{ij}f)}{\partial x_i \partial x_j}.
\end{align*}
Suppose $f>0$ and let $f = e^g$, then we have:
\begin{align*}
	e^g \frac{\partial g}{\partial t} = -e^g \sum_{i=1}^n \left( \frac{\partial \mu_i}{\partial x_i} + \mu_i \frac{\partial g}{\partial x_i} \right) + e^g \sum_{i,j=1}^n \left( D_{ij}\left( \frac{\partial^2 g}{\partial x_i \partial x_j} + \frac{\partial g}{\partial x_i} \frac{\partial g}{\partial x_j} \right) + \frac{\partial^2 D_{ij}}{\partial x_i \partial x_j} + \frac{\partial g}{\partial x_i} \frac{\partial D_{ij}}{\partial x_j} + \frac{\partial D_{ij}}{\partial x_i} \frac{\partial g}{\partial x_j} \right).
\end{align*}
So $g$ satisfies the following partial differential equation:
\begin{align*}
	\frac{\partial g}{\partial t} = -\sum_{i=1}^n \left( \frac{\partial \mu_i}{\partial x_i} + \mu_i \frac{\partial g}{\partial x_i} \right) + \sum_{i,j=1}^n \left( D_{ij}\left( \frac{\partial^2 g}{\partial x_i \partial x_j} + \frac{\partial g}{\partial x_i} \frac{\partial g}{\partial x_j} \right) + \frac{\partial^2 D_{ij}}{\partial x_i \partial x_j} + \frac{\partial g}{\partial x_i} \frac{\partial D_{ij}}{\partial x_j} + \frac{\partial D_{ij}}{\partial x_i} \frac{\partial g}{\partial x_j} \right)
\end{align*}

\section{Pendulum Example}

The Fokker-Planck equation for pendulum is
\begin{align*}
	\frac{\partial f(R,\Omega,t)}{\partial t} = -\sum_{i=1}^{3} \liediff_i (\Omega_if(R,\Omega,t)) + \sum_{i=1}^{3} \frac{\partial}{\partial \Omega_i} \left[(\Omega\times J\Omega + mg\rho\times R^Te_3)_i f(R,\Omega,t)\right] + \sum_{i,j=1}^{3} G_{ij} \frac{\partial^2 f(R,\Omega,t)}{\partial \Omega_i \partial \Omega_j}.
\end{align*}
Let $f = e^g$, then $g$ satisfies the following partial differential equation
\begin{align*}
	\frac{\partial g(R,\Omega,t)}{\partial t} = &-\sum_{i=1}^3 \Omega_i \liediff_ig(R,\Omega,t) + \sum_{i=1}^3 \frac{\partial (\Omega\times J\Omega)_i}{\partial \Omega_i} + \sum_{i=1}^3 (\Omega\times J\Omega + mg\rho\times R^Te_3)_i \frac{\partial g(R,\Omega,t)}{\partial \Omega_i} \\
	&+ \sum_{i,j=1}^3 G_{ij} \left( \frac{\partial^2 g(R,\Omega,t)}{\partial\Omega_i \partial\Omega_j} + \frac{\partial g(R,\Omega,t)}{\partial\Omega_i} \frac{\partial g(R,\Omega,t)}{\partial\Omega_j}\right).
\end{align*}

\end{document}

